% Options for packages loaded elsewhere
\PassOptionsToPackage{unicode}{hyperref}
\PassOptionsToPackage{hyphens}{url}
\PassOptionsToPackage{dvipsnames,svgnames,x11names}{xcolor}
%
\documentclass[
  letterpaper,
  DIV=11,
  numbers=noendperiod]{scrreprt}

\usepackage{amsmath,amssymb}
\usepackage{iftex}
\ifPDFTeX
  \usepackage[T1]{fontenc}
  \usepackage[utf8]{inputenc}
  \usepackage{textcomp} % provide euro and other symbols
\else % if luatex or xetex
  \usepackage{unicode-math}
  \defaultfontfeatures{Scale=MatchLowercase}
  \defaultfontfeatures[\rmfamily]{Ligatures=TeX,Scale=1}
\fi
\usepackage{lmodern}
\ifPDFTeX\else  
    % xetex/luatex font selection
\fi
% Use upquote if available, for straight quotes in verbatim environments
\IfFileExists{upquote.sty}{\usepackage{upquote}}{}
\IfFileExists{microtype.sty}{% use microtype if available
  \usepackage[]{microtype}
  \UseMicrotypeSet[protrusion]{basicmath} % disable protrusion for tt fonts
}{}
\makeatletter
\@ifundefined{KOMAClassName}{% if non-KOMA class
  \IfFileExists{parskip.sty}{%
    \usepackage{parskip}
  }{% else
    \setlength{\parindent}{0pt}
    \setlength{\parskip}{6pt plus 2pt minus 1pt}}
}{% if KOMA class
  \KOMAoptions{parskip=half}}
\makeatother
\usepackage{xcolor}
\setlength{\emergencystretch}{3em} % prevent overfull lines
\setcounter{secnumdepth}{5}
% Make \paragraph and \subparagraph free-standing
\makeatletter
\ifx\paragraph\undefined\else
  \let\oldparagraph\paragraph
  \renewcommand{\paragraph}{
    \@ifstar
      \xxxParagraphStar
      \xxxParagraphNoStar
  }
  \newcommand{\xxxParagraphStar}[1]{\oldparagraph*{#1}\mbox{}}
  \newcommand{\xxxParagraphNoStar}[1]{\oldparagraph{#1}\mbox{}}
\fi
\ifx\subparagraph\undefined\else
  \let\oldsubparagraph\subparagraph
  \renewcommand{\subparagraph}{
    \@ifstar
      \xxxSubParagraphStar
      \xxxSubParagraphNoStar
  }
  \newcommand{\xxxSubParagraphStar}[1]{\oldsubparagraph*{#1}\mbox{}}
  \newcommand{\xxxSubParagraphNoStar}[1]{\oldsubparagraph{#1}\mbox{}}
\fi
\makeatother

\usepackage{color}
\usepackage{fancyvrb}
\newcommand{\VerbBar}{|}
\newcommand{\VERB}{\Verb[commandchars=\\\{\}]}
\DefineVerbatimEnvironment{Highlighting}{Verbatim}{commandchars=\\\{\}}
% Add ',fontsize=\small' for more characters per line
\usepackage{framed}
\definecolor{shadecolor}{RGB}{241,243,245}
\newenvironment{Shaded}{\begin{snugshade}}{\end{snugshade}}
\newcommand{\AlertTok}[1]{\textcolor[rgb]{0.68,0.00,0.00}{#1}}
\newcommand{\AnnotationTok}[1]{\textcolor[rgb]{0.37,0.37,0.37}{#1}}
\newcommand{\AttributeTok}[1]{\textcolor[rgb]{0.40,0.45,0.13}{#1}}
\newcommand{\BaseNTok}[1]{\textcolor[rgb]{0.68,0.00,0.00}{#1}}
\newcommand{\BuiltInTok}[1]{\textcolor[rgb]{0.00,0.23,0.31}{#1}}
\newcommand{\CharTok}[1]{\textcolor[rgb]{0.13,0.47,0.30}{#1}}
\newcommand{\CommentTok}[1]{\textcolor[rgb]{0.37,0.37,0.37}{#1}}
\newcommand{\CommentVarTok}[1]{\textcolor[rgb]{0.37,0.37,0.37}{\textit{#1}}}
\newcommand{\ConstantTok}[1]{\textcolor[rgb]{0.56,0.35,0.01}{#1}}
\newcommand{\ControlFlowTok}[1]{\textcolor[rgb]{0.00,0.23,0.31}{\textbf{#1}}}
\newcommand{\DataTypeTok}[1]{\textcolor[rgb]{0.68,0.00,0.00}{#1}}
\newcommand{\DecValTok}[1]{\textcolor[rgb]{0.68,0.00,0.00}{#1}}
\newcommand{\DocumentationTok}[1]{\textcolor[rgb]{0.37,0.37,0.37}{\textit{#1}}}
\newcommand{\ErrorTok}[1]{\textcolor[rgb]{0.68,0.00,0.00}{#1}}
\newcommand{\ExtensionTok}[1]{\textcolor[rgb]{0.00,0.23,0.31}{#1}}
\newcommand{\FloatTok}[1]{\textcolor[rgb]{0.68,0.00,0.00}{#1}}
\newcommand{\FunctionTok}[1]{\textcolor[rgb]{0.28,0.35,0.67}{#1}}
\newcommand{\ImportTok}[1]{\textcolor[rgb]{0.00,0.46,0.62}{#1}}
\newcommand{\InformationTok}[1]{\textcolor[rgb]{0.37,0.37,0.37}{#1}}
\newcommand{\KeywordTok}[1]{\textcolor[rgb]{0.00,0.23,0.31}{\textbf{#1}}}
\newcommand{\NormalTok}[1]{\textcolor[rgb]{0.00,0.23,0.31}{#1}}
\newcommand{\OperatorTok}[1]{\textcolor[rgb]{0.37,0.37,0.37}{#1}}
\newcommand{\OtherTok}[1]{\textcolor[rgb]{0.00,0.23,0.31}{#1}}
\newcommand{\PreprocessorTok}[1]{\textcolor[rgb]{0.68,0.00,0.00}{#1}}
\newcommand{\RegionMarkerTok}[1]{\textcolor[rgb]{0.00,0.23,0.31}{#1}}
\newcommand{\SpecialCharTok}[1]{\textcolor[rgb]{0.37,0.37,0.37}{#1}}
\newcommand{\SpecialStringTok}[1]{\textcolor[rgb]{0.13,0.47,0.30}{#1}}
\newcommand{\StringTok}[1]{\textcolor[rgb]{0.13,0.47,0.30}{#1}}
\newcommand{\VariableTok}[1]{\textcolor[rgb]{0.07,0.07,0.07}{#1}}
\newcommand{\VerbatimStringTok}[1]{\textcolor[rgb]{0.13,0.47,0.30}{#1}}
\newcommand{\WarningTok}[1]{\textcolor[rgb]{0.37,0.37,0.37}{\textit{#1}}}

\providecommand{\tightlist}{%
  \setlength{\itemsep}{0pt}\setlength{\parskip}{0pt}}\usepackage{longtable,booktabs,array}
\usepackage{calc} % for calculating minipage widths
% Correct order of tables after \paragraph or \subparagraph
\usepackage{etoolbox}
\makeatletter
\patchcmd\longtable{\par}{\if@noskipsec\mbox{}\fi\par}{}{}
\makeatother
% Allow footnotes in longtable head/foot
\IfFileExists{footnotehyper.sty}{\usepackage{footnotehyper}}{\usepackage{footnote}}
\makesavenoteenv{longtable}
\usepackage{graphicx}
\makeatletter
\newsavebox\pandoc@box
\newcommand*\pandocbounded[1]{% scales image to fit in text height/width
  \sbox\pandoc@box{#1}%
  \Gscale@div\@tempa{\textheight}{\dimexpr\ht\pandoc@box+\dp\pandoc@box\relax}%
  \Gscale@div\@tempb{\linewidth}{\wd\pandoc@box}%
  \ifdim\@tempb\p@<\@tempa\p@\let\@tempa\@tempb\fi% select the smaller of both
  \ifdim\@tempa\p@<\p@\scalebox{\@tempa}{\usebox\pandoc@box}%
  \else\usebox{\pandoc@box}%
  \fi%
}
% Set default figure placement to htbp
\def\fps@figure{htbp}
\makeatother

\KOMAoption{captions}{tableheading}
\makeatletter
\@ifpackageloaded{bookmark}{}{\usepackage{bookmark}}
\makeatother
\makeatletter
\@ifpackageloaded{caption}{}{\usepackage{caption}}
\AtBeginDocument{%
\ifdefined\contentsname
  \renewcommand*\contentsname{Table of contents}
\else
  \newcommand\contentsname{Table of contents}
\fi
\ifdefined\listfigurename
  \renewcommand*\listfigurename{List of Figures}
\else
  \newcommand\listfigurename{List of Figures}
\fi
\ifdefined\listtablename
  \renewcommand*\listtablename{List of Tables}
\else
  \newcommand\listtablename{List of Tables}
\fi
\ifdefined\figurename
  \renewcommand*\figurename{Figure}
\else
  \newcommand\figurename{Figure}
\fi
\ifdefined\tablename
  \renewcommand*\tablename{Table}
\else
  \newcommand\tablename{Table}
\fi
}
\@ifpackageloaded{float}{}{\usepackage{float}}
\floatstyle{ruled}
\@ifundefined{c@chapter}{\newfloat{codelisting}{h}{lop}}{\newfloat{codelisting}{h}{lop}[chapter]}
\floatname{codelisting}{Listing}
\newcommand*\listoflistings{\listof{codelisting}{List of Listings}}
\makeatother
\makeatletter
\makeatother
\makeatletter
\@ifpackageloaded{caption}{}{\usepackage{caption}}
\@ifpackageloaded{subcaption}{}{\usepackage{subcaption}}
\makeatother

\usepackage{bookmark}

\IfFileExists{xurl.sty}{\usepackage{xurl}}{} % add URL line breaks if available
\urlstyle{same} % disable monospaced font for URLs
\hypersetup{
  pdftitle={MA500: Introduction to R},
  colorlinks=true,
  linkcolor={blue},
  filecolor={Maroon},
  citecolor={Blue},
  urlcolor={Blue},
  pdfcreator={LaTeX via pandoc}}


\title{MA500: Introduction to R}
\author{}
\date{}

\begin{document}
\maketitle

\renewcommand*\contentsname{Table of contents}
{
\hypersetup{linkcolor=}
\setcounter{tocdepth}{2}
\tableofcontents
}

\bookmarksetup{startatroot}

\chapter*{MA-500: Introduction to R}\label{ma-500-introduction-to-r}
\addcontentsline{toc}{chapter}{MA-500: Introduction to R}

\markboth{MA-500: Introduction to R}{MA-500: Introduction to R}

FANUCHÅNAN 2025: 8/20/2025 - 10/11/2025

\textbf{Instructor Information}

Regina-Mae Dominguez \textbar{} dominguezr@triton.uog.edu

\textbf{Office Hours \& Location:} Online by Appointments

\section*{Course Introduction}\label{course-introduction}
\addcontentsline{toc}{section}{Course Introduction}

\markright{Course Introduction}

This \texttt{markdown} site will include notes, guides, and resources,
while Moodle will be the central hub for the course as well as the
platform for uploading and submitting homework assignments. As the
course progresses, this site will be regularly updated to ensure it's
helpful, easy to navigate, and accessible whenever you need a reference.
While this resource is designed to be useful, I \textbf{\emph{strongly}}
encourage you to take your own notes for a more personalized and
detailed study guide.

\bookmarksetup{startatroot}

\chapter*{R/RStudio Installation
Guide}\label{rrstudio-installation-guide}
\addcontentsline{toc}{chapter}{R/RStudio Installation Guide}

\markboth{R/RStudio Installation Guide}{R/RStudio Installation Guide}

\section*{Installing R}\label{installing-r}
\addcontentsline{toc}{section}{Installing R}

\markright{Installing R}

To install R, begin by visiting the Comprehensive R Archive Network
(CRAN) here: \url{https://cran.r-project.org/.} Select and download the
appropriate R binary package for your operating system--- whether
Windows, macOS, or Linux. For Mac users, please be sure that you install
the correct package binary associated with your processor (e.g., Intel
or Apple Silicon).

\section*{Installing RStudio}\label{installing-rstudio}
\addcontentsline{toc}{section}{Installing RStudio}

\markright{Installing RStudio}

RStudio is the most widely used integrated development environment (IDE)
for R programming. You can download the free version here:
\url{https://posit.co/downloads/.} Whie you have the option to use
alternative IDEs, such as VS Code with the Rtools extention or the base
R GUI, it is recommended to use RStudio as the course material will
primarily be demonstrated using this IDE. This will ensure you can
easily follow along with the course content!

\bookmarksetup{startatroot}

\chapter*{R Basics \& Fundamentals}\label{r-basics-fundamentals}
\addcontentsline{toc}{chapter}{R Basics \& Fundamentals}

\markboth{R Basics \& Fundamentals}{R Basics \& Fundamentals}

\section*{R Markdown}\label{r-markdown}
\addcontentsline{toc}{section}{R Markdown}

\markright{R Markdown}

This is an R Markdown document. Markdown is a simple formatting syntax
for authoring HTML, PDF, and MS Word documents. For more details on
using R Markdown see \url{http://rmarkdown.rstudio.com}.

When you click the \textbf{Knit} button a document will be generated
that includes both content as well as the output of any embedded R code
chunks within the document. You can embed an R code chunk like this:

Important commands when using RMarkdown: Use the \texttt{/} to quick
insert markdown elements: R code chunk, Heading/Format Options, etc.

In-line code notation: ` \textless- symbol with \textasciitilde{} on
keyboard?

\section*{Let's move onto R Basics!}\label{lets-move-onto-r-basics}
\addcontentsline{toc}{section}{Let's move onto R Basics!}

\markright{Let's move onto R Basics!}

\section*{Comments}\label{comments}
\addcontentsline{toc}{section}{Comments}

\markright{Comments}

It is recommended to always comment/annotate your code to make concepts
clear. In R, we use the \texttt{\#} symbol to indicate a comment.

\begin{Shaded}
\begin{Highlighting}[]
\CommentTok{\# this is a comment }

\CommentTok{\#\textquotesingle{} }
\CommentTok{\#\textquotesingle{} this is a multi{-}line comment}
\CommentTok{\#\textquotesingle{} }
\end{Highlighting}
\end{Shaded}

\section*{Using R as a calculator}\label{using-r-as-a-calculator}
\addcontentsline{toc}{section}{Using R as a calculator}

\markright{Using R as a calculator}

\begin{Shaded}
\begin{Highlighting}[]
\CommentTok{\# in either a script or the console, you can use it to solve basic arithmentic}
\DecValTok{1} \SpecialCharTok{/} \DecValTok{200} \SpecialCharTok{*} \DecValTok{30}
\end{Highlighting}
\end{Shaded}

\begin{verbatim}
[1] 0.15
\end{verbatim}

\begin{Shaded}
\begin{Highlighting}[]
\FunctionTok{sin}\NormalTok{(pi}\SpecialCharTok{/}\DecValTok{2}\NormalTok{)}
\end{Highlighting}
\end{Shaded}

\begin{verbatim}
[1] 1
\end{verbatim}

\section*{R Packages}\label{r-packages}
\addcontentsline{toc}{section}{R Packages}

\markright{R Packages}

The \texttt{base} installation of R consists of many in-house
\texttt{functions} and \texttt{commands}, but more specialized
techniques would require the installation of packages. A few packages
that you should have for this course and I would recommend downloading
are:

\begin{itemize}
\item
  \texttt{ggplot2}
\item
  \texttt{MASS}
\item
  \texttt{stats}
\item
  \texttt{tidyverse}/\texttt{dplyr}
\end{itemize}

\emph{Note:} If you are using a mac, some development packages would
require you to install \texttt{XQuartz} or \texttt{XCode}. If issues
come up with installing a package or getting a package to work, please
let me know so I can help you out!

Find more packages here: \url{https://cran.r-project.org/}

\begin{Shaded}
\begin{Highlighting}[]
\CommentTok{\# how to install packages}
\FunctionTok{install.packages}\NormalTok{(}\StringTok{"MASS"}\NormalTok{)}
\end{Highlighting}
\end{Shaded}

\begin{Shaded}
\begin{Highlighting}[]
\CommentTok{\# calling the package after installation}
\FunctionTok{library}\NormalTok{(}\StringTok{"MASS"}\NormalTok{)}
\end{Highlighting}
\end{Shaded}

\begin{Shaded}
\begin{Highlighting}[]
\CommentTok{\# updating packages/all packages}
\FunctionTok{update.packages}\NormalTok{()}
\end{Highlighting}
\end{Shaded}

\section*{Directories}\label{directories}
\addcontentsline{toc}{section}{Directories}

\markright{Directories}

When you run RStudio, your session is typically associated with a
\emph{working directory.} This is the default location where your files
are imported or saved.

\begin{Shaded}
\begin{Highlighting}[]
\CommentTok{\# check your working directory with this function}
\FunctionTok{getwd}\NormalTok{()}
\end{Highlighting}
\end{Shaded}

\begin{verbatim}
[1] "/Users/rdominguez/Documents/MA500/Fall25/ma500_fall2025"
\end{verbatim}

\begin{Shaded}
\begin{Highlighting}[]
\FunctionTok{setwd}\NormalTok{(}\StringTok{"/insertFilePathHere"}\NormalTok{)}
\end{Highlighting}
\end{Shaded}

\section*{Variables}\label{variables}
\addcontentsline{toc}{section}{Variables}

\markright{Variables}

You can save and create new objects to store your results by using
\texttt{\textless{}-} (as opposed to the conventional \texttt{=} which
is used with \texttt{named\ function\ assignments}.)

\begin{Shaded}
\begin{Highlighting}[]
\CommentTok{\# store the result of 4*3 into x}

\NormalTok{x }\OtherTok{\textless{}{-}} \DecValTok{4} \SpecialCharTok{*} \DecValTok{3}

\CommentTok{\# print x (assignment alone does not print output)}
\FunctionTok{print}\NormalTok{(x)}
\end{Highlighting}
\end{Shaded}

\begin{verbatim}
[1] 12
\end{verbatim}

Note: R is a \emph{dynamic language}, so the types and values can easily
be changed. (type checks are done during run-time)

\begin{Shaded}
\begin{Highlighting}[]
\CommentTok{\# x is now a of type character}
\NormalTok{x }\OtherTok{\textless{}{-}} \StringTok{"statistics"}
\end{Highlighting}
\end{Shaded}

\section*{R data types}\label{r-data-types}
\addcontentsline{toc}{section}{R data types}

\markright{R data types}

A \textbf{data type} describes the kind of values a variable can hold
and how R will interpret and use those values in calculations,
comparisons, and functions. Choosing/creating the right data type is
important because it affects how R stores information and what
operations you can perform.

R data types include:

\begin{itemize}
\item
  \textbf{\texttt{numeric}}:

  \begin{itemize}
  \item
    \texttt{double} values with decimals
  \item
    \texttt{int} whole numbers
  \end{itemize}
\item
  \textbf{\texttt{character}} -- text or strings (\texttt{"Hello"})
\item
  \textbf{\texttt{logical}} -- TRUE/FALSE values
\item
  \textbf{\texttt{factor}} -- categorical data (\texttt{"A"},
  \texttt{"B"})
\end{itemize}

Use the \texttt{class()} function to determine what data type your
variable is:

\begin{Shaded}
\begin{Highlighting}[]
\FunctionTok{class}\NormalTok{(}\DecValTok{2}\NormalTok{L)}
\end{Highlighting}
\end{Shaded}

\begin{verbatim}
[1] "integer"
\end{verbatim}

\begin{Shaded}
\begin{Highlighting}[]
\FunctionTok{class}\NormalTok{(}\DecValTok{2}\NormalTok{)}
\end{Highlighting}
\end{Shaded}

\begin{verbatim}
[1] "numeric"
\end{verbatim}

\section*{Data Vectors}\label{data-vectors}
\addcontentsline{toc}{section}{Data Vectors}

\markright{Data Vectors}

An \textbf{R data vector} is a collection of observations or
measurements concerning a single variable \textbf{of the same data type}
(all numeric, all character, all logical, etc.). The \texttt{c()}
function takes individual values and \emph{combines them into a single
vector}.

\subsection*{Example}\label{example}
\addcontentsline{toc}{subsection}{Example}

In 2021, the average temperature in Guam for each month was 80.7, 81.4,
81.5, 82.6, 82.6, 83.7, 83.4, 81.8, 82.5, 81.4, 82.0, and 81.0. Store
these values in a vector named \texttt{temp}.

\emph{Solution:}

\begin{Shaded}
\begin{Highlighting}[]
\NormalTok{temp }\OtherTok{\textless{}{-}} \FunctionTok{c}\NormalTok{(}\FloatTok{80.7}\NormalTok{, }\FloatTok{81.4}\NormalTok{, }\FloatTok{81.5}\NormalTok{, }\FloatTok{82.6}\NormalTok{, }\FloatTok{82.6}\NormalTok{, }\FloatTok{83.7}\NormalTok{, }\FloatTok{83.4}\NormalTok{, }\FloatTok{81.8}\NormalTok{, }\FloatTok{82.5}\NormalTok{, }\FloatTok{81.4}\NormalTok{, }\FloatTok{82.0}\NormalTok{, }\FloatTok{81.0}\NormalTok{)}
\end{Highlighting}
\end{Shaded}

\begin{Shaded}
\begin{Highlighting}[]
\CommentTok{\# returns the length of temp }
\FunctionTok{length}\NormalTok{(temp)}
\end{Highlighting}
\end{Shaded}

\begin{verbatim}
[1] 12
\end{verbatim}

\begin{Shaded}
\begin{Highlighting}[]
\CommentTok{\# we can perform these function because all values in temp is a numeric value}
\FunctionTok{sum}\NormalTok{(temp)}
\end{Highlighting}
\end{Shaded}

\begin{verbatim}
[1] 984.6
\end{verbatim}

\begin{Shaded}
\begin{Highlighting}[]
\FunctionTok{mean}\NormalTok{(temp)}
\end{Highlighting}
\end{Shaded}

\begin{verbatim}
[1] 82.05
\end{verbatim}

In the case you have a \texttt{NA} in your vector, \texttt{sum()} and
\texttt{mean()} will return \texttt{NA}. To bypass that, you need to
include the argument: \texttt{na.rm\ =\ TRUE}.

\subsection*{Attributes}\label{attributes}
\addcontentsline{toc}{subsection}{Attributes}

Vectors can be assigned \textbf{attributes} or \textbf{names.}

\begin{Shaded}
\begin{Highlighting}[]
\NormalTok{months }\OtherTok{\textless{}{-}} \FunctionTok{c}\NormalTok{(}\StringTok{"Jan"}\NormalTok{, }\StringTok{"Feb"}\NormalTok{, }\StringTok{"Mar"}\NormalTok{, }\StringTok{"Apr"}\NormalTok{, }\StringTok{"May"}\NormalTok{, }\StringTok{"Jun"}\NormalTok{, }\StringTok{"July"}\NormalTok{,}
    \StringTok{"Aug"}\NormalTok{, }\StringTok{"Sep"}\NormalTok{, }\StringTok{"Oct"}\NormalTok{, }\StringTok{"Nov"}\NormalTok{, }\StringTok{"Dec"}\NormalTok{)}

\FunctionTok{names}\NormalTok{(temp) }\OtherTok{\textless{}{-}}\NormalTok{ months}
\FunctionTok{print}\NormalTok{(temp)}
\end{Highlighting}
\end{Shaded}

\begin{verbatim}
 Jan  Feb  Mar  Apr  May  Jun July  Aug  Sep  Oct  Nov  Dec 
80.7 81.4 81.5 82.6 82.6 83.7 83.4 81.8 82.5 81.4 82.0 81.0 
\end{verbatim}

\subsection*{Indexing}\label{indexing}
\addcontentsline{toc}{subsection}{Indexing}

You can call a specific value in the \texttt{temp} vector by referencing
its respective name.

\begin{Shaded}
\begin{Highlighting}[]
\CommentTok{\# get the average temperature in May}
\NormalTok{temp[}\StringTok{"May"}\NormalTok{]}
\end{Highlighting}
\end{Shaded}

\begin{verbatim}
 May 
82.6 
\end{verbatim}

OR, you can index the vector by referencing its numeric index value.

\begin{Shaded}
\begin{Highlighting}[]
\CommentTok{\# grabs the first value}
\NormalTok{temp[}\DecValTok{1}\NormalTok{]}
\end{Highlighting}
\end{Shaded}

\begin{verbatim}
 Jan 
80.7 
\end{verbatim}

\begin{Shaded}
\begin{Highlighting}[]
\CommentTok{\# grab the first 4 values}
\NormalTok{temp[}\DecValTok{1}\SpecialCharTok{:}\DecValTok{4}\NormalTok{]}
\end{Highlighting}
\end{Shaded}

\begin{verbatim}
 Jan  Feb  Mar  Apr 
80.7 81.4 81.5 82.6 
\end{verbatim}

\subsection*{Operators}\label{operators}
\addcontentsline{toc}{subsection}{Operators}

\textbf{Operators} are symbols that tell R what kind of computation to
perform on values or variables. They are the building blocks for
calculations, comparisons, and logical expressions.

\subsubsection*{Comparison:}\label{comparison}
\addcontentsline{toc}{subsubsection}{Comparison:}

\begin{itemize}
\tightlist
\item
  \texttt{\textgreater{}} greater than
\item
  \texttt{\textless{}} less than
\item
  \texttt{\textless{}=} less than or equal to
\item
  \texttt{\textgreater{}=} greater than or equal to
\item
  \texttt{==} equal to
\item
  \texttt{!=} not equal to
\end{itemize}

\subsection*{Example:}\label{example-1}
\addcontentsline{toc}{subsection}{Example:}

What months have a higher average temperature compared to August?

\emph{Solution:}

\begin{Shaded}
\begin{Highlighting}[]
\NormalTok{temp[temp }\SpecialCharTok{\textless{}}\NormalTok{ temp[}\StringTok{"Aug"}\NormalTok{]]}
\end{Highlighting}
\end{Shaded}

\begin{verbatim}
 Jan  Feb  Mar  Oct  Dec 
80.7 81.4 81.5 81.4 81.0 
\end{verbatim}

Let's break the line of code down:

\begin{itemize}
\item
  \texttt{temp{[}\textquotesingle{}Aug\textquotesingle{}{]}} : refers to
  the temperature in August.
\item
  \texttt{temp\ \textgreater{}\ temp{[}\textquotesingle{}Aug\textquotesingle{}{]}}
  : checks all values of the \texttt{temp} vector to check if this
  statement is true
\item
  \texttt{temp{[}temp\ \textgreater{}\ temp{[}\textquotesingle{}Aug\textquotesingle{}{]}{]}}:
  returns only the months \& values where the second bullet is true.
\end{itemize}

\subsubsection*{Logical}\label{logical}
\addcontentsline{toc}{subsubsection}{Logical}

\begin{itemize}
\tightlist
\item
  \texttt{\&} AND
\item
  \texttt{\textbar{}} OR
\item
  \texttt{!} NOT
\end{itemize}

\subsection*{Example - using or
operator:}\label{example---using-or-operator}
\addcontentsline{toc}{subsection}{Example - using or operator:}

What FALL months have a higher average temperature than August?

\emph{Solution:}

\begin{Shaded}
\begin{Highlighting}[]
\CommentTok{\# let\textquotesingle{}s define a vector that contains the fall months }

\NormalTok{fall }\OtherTok{\textless{}{-}} \FunctionTok{c}\NormalTok{(}\StringTok{"Sep"}\NormalTok{, }\StringTok{"Oct"}\NormalTok{, }\StringTok{"Nov"}\NormalTok{)}
\NormalTok{temp[temp }\SpecialCharTok{\textgreater{}}\NormalTok{ temp[}\StringTok{"Aug"}\NormalTok{] }\SpecialCharTok{|} \FunctionTok{names}\NormalTok{(temp) }\SpecialCharTok{\%in\%}\NormalTok{ fall]}
\end{Highlighting}
\end{Shaded}

\begin{verbatim}
 Apr  May  Jun July  Sep  Oct  Nov 
82.6 82.6 83.7 83.4 82.5 81.4 82.0 
\end{verbatim}

\begin{itemize}
\item
  \texttt{names(temp)} : calls the attributes of the \texttt{temp}
  vector
\item
  \texttt{\%in\%} : a special operator that checks if elements exist in
  a vector
\item
  \texttt{names(temp)\ \%in\%\ fall} : checks which attributes are in
  the \texttt{fall} vector
\end{itemize}

\subsection*{Example - using equals
operator:}\label{example---using-equals-operator}
\addcontentsline{toc}{subsection}{Example - using equals operator:}

What month had the highest average temperature?

\emph{Solution:}

\begin{Shaded}
\begin{Highlighting}[]
\NormalTok{temp[temp }\SpecialCharTok{==} \FunctionTok{max}\NormalTok{(temp)]}
\end{Highlighting}
\end{Shaded}

\begin{verbatim}
 Jun 
83.7 
\end{verbatim}

OR if we want just the name of the month, we can index the value from
the names attribute:

\begin{Shaded}
\begin{Highlighting}[]
\FunctionTok{names}\NormalTok{(temp)[temp }\SpecialCharTok{==} \FunctionTok{max}\NormalTok{(temp)]}
\end{Highlighting}
\end{Shaded}

\begin{verbatim}
[1] "Jun"
\end{verbatim}

\subsection*{Example}\label{example-2}
\addcontentsline{toc}{subsection}{Example}

What is the average temperature in the summer months?

\emph{Solution:}

\begin{Shaded}
\begin{Highlighting}[]
\NormalTok{temp[}\FunctionTok{c}\NormalTok{(}\StringTok{"Jun"}\NormalTok{, }\StringTok{"July"}\NormalTok{, }\StringTok{"Aug"}\NormalTok{)]}
\end{Highlighting}
\end{Shaded}

\begin{verbatim}
 Jun July  Aug 
83.7 83.4 81.8 
\end{verbatim}

\begin{Shaded}
\begin{Highlighting}[]
\NormalTok{summer\_months }\OtherTok{\textless{}{-}} \FunctionTok{c}\NormalTok{(}\StringTok{"Jun"}\NormalTok{, }\StringTok{"July"}\NormalTok{, }\StringTok{"Aug"}\NormalTok{)}
\NormalTok{summer\_temps }\OtherTok{\textless{}{-}}\NormalTok{ temp[summer\_months]}

\NormalTok{average\_summer }\OtherTok{\textless{}{-}} \FunctionTok{mean}\NormalTok{(summer\_temps)}
\FunctionTok{print}\NormalTok{(average\_summer)}
\end{Highlighting}
\end{Shaded}

\begin{verbatim}
[1] 82.96667
\end{verbatim}

\section*{Vector Operations}\label{vector-operations}
\addcontentsline{toc}{section}{Vector Operations}

\markright{Vector Operations}

With vectors, you can utilize element-wise basic arithmetic.

For example, if we want to add 2 degrees to every temperature in the
\texttt{temp} data vector, then we can define that as:

\begin{Shaded}
\begin{Highlighting}[]
\NormalTok{temp }\OtherTok{\textless{}{-}}\NormalTok{ temp }\SpecialCharTok{+} \DecValTok{2}
\end{Highlighting}
\end{Shaded}

\subsection*{Example - Vector
Operation}\label{example---vector-operation}
\addcontentsline{toc}{subsection}{Example - Vector Operation}

Convert all temperatures in our \texttt{temp} vector from Fahrenheit to
Celsius.

We are going to apply this formula:

\[
C = \frac{5}{9}*(t - 32)
\]

to all values in our \texttt{temp} vector.

\emph{Solution:}

\begin{Shaded}
\begin{Highlighting}[]
\NormalTok{temp\_celcius }\OtherTok{\textless{}{-}}\NormalTok{ (}\DecValTok{5}\SpecialCharTok{/}\DecValTok{9}\NormalTok{) }\SpecialCharTok{*}\NormalTok{ (temp }\SpecialCharTok{{-}} \DecValTok{32}\NormalTok{)}
\end{Highlighting}
\end{Shaded}

All attributes are retained and all values are converted.

\begin{verbatim}
Example: 

Convert all temperatures from Fahrenheit to Celciu
\end{verbatim}

\section*{Sequences and Repetition}\label{sequences-and-repetition}
\addcontentsline{toc}{section}{Sequences and Repetition}

\markright{Sequences and Repetition}

\texttt{R} has in-house functions that allow you to create a sequence of
values (or repeat).

We will be using \texttt{seq} and \texttt{rep}, and you can read up for
help on the documentation by using ?function in R. This pulls up the
documentation and example usage

\subsection*{Sequence}\label{sequence}
\addcontentsline{toc}{subsection}{Sequence}

Without a function, the easiest way to create a sequence with intervals
of 1 is using the colon operator

\begin{Shaded}
\begin{Highlighting}[]
\FunctionTok{print}\NormalTok{(}\DecValTok{1}\SpecialCharTok{:}\DecValTok{10}\NormalTok{)}
\end{Highlighting}
\end{Shaded}

\begin{verbatim}
 [1]  1  2  3  4  5  6  7  8  9 10
\end{verbatim}

For more flexible sequences, we can use the \texttt{seq} function:

\subsection*{Example - Sequence List}\label{example---sequence-list}
\addcontentsline{toc}{subsection}{Example - Sequence List}

Create a list of every 3rd value from \texttt{1} to \texttt{20}

\begin{Shaded}
\begin{Highlighting}[]
\FunctionTok{seq}\NormalTok{(}\AttributeTok{from =} \DecValTok{1}\NormalTok{, }\AttributeTok{to =} \DecValTok{20}\NormalTok{, }\AttributeTok{by =} \DecValTok{3}\NormalTok{)}
\end{Highlighting}
\end{Shaded}

\begin{verbatim}
[1]  1  4  7 10 13 16 19
\end{verbatim}

Note: This will \emph{always} include the \texttt{from} value, but NOT
the \texttt{to} value. This is dependent on the \texttt{by} number.

Instead of specifying a \texttt{by} number, you can specify how many
numbers you would like in between with \texttt{length.out} - this would
be equally spaced.

\begin{Shaded}
\begin{Highlighting}[]
\FunctionTok{seq}\NormalTok{(}\AttributeTok{from =} \DecValTok{1}\NormalTok{, }\AttributeTok{to =} \DecValTok{10}\NormalTok{, }\AttributeTok{length.out =} \DecValTok{20}\NormalTok{)}
\end{Highlighting}
\end{Shaded}

\begin{verbatim}
 [1]  1.000000  1.473684  1.947368  2.421053  2.894737  3.368421  3.842105
 [8]  4.315789  4.789474  5.263158  5.736842  6.210526  6.684211  7.157895
[15]  7.631579  8.105263  8.578947  9.052632  9.526316 10.000000
\end{verbatim}

For decreasing, set \texttt{\textasciigrave{}by\textasciigrave{}} to be
a negative value and switch the \texttt{from} and \texttt{to} values.

\begin{Shaded}
\begin{Highlighting}[]
\FunctionTok{seq}\NormalTok{(}\AttributeTok{from =} \DecValTok{10}\NormalTok{, }\AttributeTok{to =} \DecValTok{1}\NormalTok{, }\AttributeTok{by =} \SpecialCharTok{{-}}\DecValTok{3}\NormalTok{)}
\end{Highlighting}
\end{Shaded}

\begin{verbatim}
[1] 10  7  4  1
\end{verbatim}

\section*{Random Sampling}\label{random-sampling}
\addcontentsline{toc}{section}{Random Sampling}

\markright{Random Sampling}

The \href{https://rdrr.io/r/base/sample.html}{\texttt{sample()}}
function takes a sample from the specified elements of \texttt{x} with
or without replacement.

\subsection*{Example - Creating a
sample}\label{example---creating-a-sample}
\addcontentsline{toc}{subsection}{Example - Creating a sample}

Create a random sample of \texttt{20} values ranging from \texttt{1} to
\texttt{100} with replacement.

\begin{Shaded}
\begin{Highlighting}[]
\FunctionTok{sample}\NormalTok{(}\DecValTok{1}\SpecialCharTok{:}\DecValTok{100}\NormalTok{, }\DecValTok{20}\NormalTok{, }\AttributeTok{replace =}\NormalTok{ T)}
\end{Highlighting}
\end{Shaded}

\begin{verbatim}
 [1] 95 69 44 29  2 97  6 87  2 77 85 55 39 36 19 54 61 93 69 70
\end{verbatim}

Note: running the above code again will not give you the same output:

\begin{Shaded}
\begin{Highlighting}[]
\FunctionTok{sample}\NormalTok{(}\DecValTok{1}\SpecialCharTok{:}\DecValTok{100}\NormalTok{, }\DecValTok{20}\NormalTok{, }\AttributeTok{replace =}\NormalTok{ T)}
\end{Highlighting}
\end{Shaded}

\begin{verbatim}
 [1] 32 93 39 42 67 48 28 68 92 62 31 41 61 83 87  5 98 27 43 52
\end{verbatim}

If you would like to replicate the same results every time, we have to
set a seed by using the \texttt{set.seed(x)} function, where \texttt{x}
is any arbitrary number.

\begin{Shaded}
\begin{Highlighting}[]
\FunctionTok{set.seed}\NormalTok{(}\DecValTok{415}\NormalTok{)}
\FunctionTok{sample}\NormalTok{(}\DecValTok{1}\SpecialCharTok{:}\DecValTok{100}\NormalTok{, }\DecValTok{20}\NormalTok{, }\AttributeTok{replace =}\NormalTok{ T)}
\end{Highlighting}
\end{Shaded}

\begin{verbatim}
 [1]  2 94 70 11 91 35 56 47 34 39  1 18 60 29 68  9 70 45 53 67
\end{verbatim}

\subsection*{Repetition}\label{repetition}
\addcontentsline{toc}{subsection}{Repetition}

To repeat values in a vector, use the \texttt{rep()} function.

\begin{Shaded}
\begin{Highlighting}[]
\CommentTok{\# repeat 1 four times }
\FunctionTok{rep}\NormalTok{(}\AttributeTok{x =} \DecValTok{1}\NormalTok{, }\AttributeTok{times =} \DecValTok{4}\NormalTok{)}
\end{Highlighting}
\end{Shaded}

\begin{verbatim}
[1] 1 1 1 1
\end{verbatim}

\begin{Shaded}
\begin{Highlighting}[]
\CommentTok{\# repeat 1 to 5, 10 times}
\FunctionTok{rep}\NormalTok{(}\DecValTok{1}\SpecialCharTok{:}\DecValTok{5}\NormalTok{, }\AttributeTok{times =} \DecValTok{10}\NormalTok{)}
\end{Highlighting}
\end{Shaded}

\begin{verbatim}
 [1] 1 2 3 4 5 1 2 3 4 5 1 2 3 4 5 1 2 3 4 5 1 2 3 4 5 1 2 3 4 5 1 2 3 4 5 1 2 3
[39] 4 5 1 2 3 4 5 1 2 3 4 5
\end{verbatim}

\begin{Shaded}
\begin{Highlighting}[]
\CommentTok{\# repeat 1, 2, 5, 7 each values twice}
\FunctionTok{rep}\NormalTok{(}\AttributeTok{x =} \FunctionTok{c}\NormalTok{(}\DecValTok{1}\NormalTok{, }\DecValTok{2}\NormalTok{, }\DecValTok{5}\NormalTok{, }\DecValTok{7}\NormalTok{), }\AttributeTok{times =} \DecValTok{3}\NormalTok{, }\AttributeTok{each =} \DecValTok{2}\NormalTok{)}
\end{Highlighting}
\end{Shaded}

\begin{verbatim}
 [1] 1 1 2 2 5 5 7 7 1 1 2 2 5 5 7 7 1 1 2 2 5 5 7 7
\end{verbatim}

\begin{Shaded}
\begin{Highlighting}[]
\CommentTok{\# repeat each element in x by specified vector time (1}
\CommentTok{\# once, 2 four times, 5, once, and 7 twice)}
\FunctionTok{rep}\NormalTok{(}\AttributeTok{x =} \FunctionTok{c}\NormalTok{(}\DecValTok{1}\NormalTok{, }\DecValTok{2}\NormalTok{, }\DecValTok{5}\NormalTok{, }\DecValTok{7}\NormalTok{), }\AttributeTok{times =} \FunctionTok{c}\NormalTok{(}\DecValTok{1}\NormalTok{, }\DecValTok{4}\NormalTok{, }\DecValTok{1}\NormalTok{, }\DecValTok{2}\NormalTok{))}
\end{Highlighting}
\end{Shaded}

\begin{verbatim}
[1] 1 2 2 2 2 5 7 7
\end{verbatim}




\end{document}
